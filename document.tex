\documentclass[11pt, a4paper]{article}

\usepackage[utf8]{inputenc}
\usepackage[swedish]{babel}

\usepackage{parskip}
\usepackage{setspace}
\usepackage[babel]{microtype}
\usepackage[labelsep=period, labelfont=bf, textfont=it, skip=5pt]{caption}
\usepackage{dirtytalk}

\usepackage[style=apa, citestyle=apa]{biblatex}
\addbibresource{bibliography.bib}

\usepackage{amsmath}
\usepackage{amsfonts}

\usepackage{svg}
\svgpath{{images/}}
% To include SVG figure:
%\includesvg[inkscapelatex=false, width=\textwidth]{svg_image}

\usepackage{graphicx}
\graphicspath{{images/}}

\title{Magnetisk resonanstomografi}
\author{Björn Sundin\medskip\\\normalsize Fysik 3 - NTI Kronhus}

\begin{document}

\maketitle

\clearpage
\section{Bakgrund}

% https://www.gu.se/nmr-spektroskopi/vad-ar-nmr

Magnetisk resonanstomografi förkortas ofta MRI (Magnetic Resonance Imaging). Även kärnmagnetisk resonans/NMR (Nuclear Magnetic Resonance) används som benämning, men denna används främst för det fysikaliska fenomenet och inte i sammanhang av diagnostiska undersökningar av individer \parencite{nmr_eller_mri}. \say{Magnetisk resonans} är mindre specifikt än \say{kärnmagnetisk resonans} eftersom det även finns elektronspinnresonans som involverar elektronen och inte atomkärnan, men ordet \say{nuclear} eller prefixet \say{kärn-} förknippas bland allmänheten med farliga saker, så man undviker ofta att använda den mer specifika benämningen.

NMR används även inom kemin för att undersöka prover

\parencite{mri_lärobok}

\clearpage
\section{Teori}
Kärnan hos väteatomer består av en proton, som har spinnkvanttalet $s=\frac{1}{2}$. Spinnet gör att partikeln beter sig som en magnet med en nordpol och en sydpol. Det magnetiska spinnkvanttalet bestämmer det specifika spinntillståndet och kan ha värdena $m_s=s+n$ där $n\in\{-2s..0\}$. För 

Bara kärnor med ett nollskiljt spinn kan användas för kärnmagnetisk resonans

Spinnkvanttalet $s$ bestämmer ett rörelsemängdsmoment som är speciellt för varje typ av partikel. Endast riktningen av spinnet $m_s$ kan ändras hos en partikel, inte magnituden. Hos en elektron är $s=\frac{1}{2}$ och $m_s=\pm\frac{1}{2}$.

Precession-frekvensen för protonerna bestäms av larmor-ekvationen \parencite{larmor_frekvens}:
\begin{equation}
    \omega=2\pi f=\gamma B_0
\end{equation}

\clearpage
\section{Användningsområden}


\clearpage
\section{Avslutning}

\clearpage
\printbibliography

\end{document}
