\documentclass[11pt, a4paper]{article}
\usepackage[margin={3cm, 2.5cm}]{geometry}

\usepackage[utf8]{inputenc}
\usepackage[swedish]{babel}

\usepackage{parskip}
\usepackage{setspace}
\usepackage[babel]{microtype}
\usepackage[labelsep=period, labelfont=bf, skip=5pt]{caption}
\usepackage{subcaption}
\usepackage{dirtytalk}

\usepackage[style=apa, citestyle=apa]{biblatex}
\addbibresource{bibliography.bib}

\usepackage{amsmath}
\usepackage{amsfonts}
\usepackage{siunitx}

\usepackage{svg}
\svgpath{{images/}}
% To include SVG figure:
%\includesvg[inkscapelatex=false, width=\textwidth]{svg_image}

\usepackage{graphicx}
\graphicspath{{images/}}

\title{Magnetisk resonanstomografi}
\author{Björn Sundin\medskip\\\normalsize Fysik 3 - NTI Kronhus}

\begin{document}

\maketitle

\vfill
\begin{center}
	\includegraphics[width=0.9\textwidth]{mri_scan.jpg}	
\end{center}
\vspace{1cm}
\vfill

\clearpage
\begin{center}
	\textbf{Sammanfattning}
\end{center}

Syftet med denna text är att förklara fysiken bakom magnetkameran. Målgruppen är elever som har gått gymnasiekursen Fysik 3. Först ges en historisk bakgrund och introduktion till ämnet, samt en beskrivning av MRI-systemets beståndsdelar. Därefter introduceras koncepten av spinn och precession hos subatomära partiklar. Dessa kvantmekaniska koncept sätts sedan in i sitt sammanhang med magnetkameran. Förklaringen av magnetkamerans funktion omfattar de olika fysikaliska processerna som sker i tur och ordning, hur data samlas in och tolkas, och till sist en kortfattad beskrivning av avbildningstekniken. Vissa kringliggande fysikaliska samband tas upp genom texten för att ge en bättre förståelse.
\vspace{0.5cm}
\tableofcontents

\clearpage
\section{Bakgrund}

Magnetisk resonanstomografi, förkortat MRT eller MRI (Magnetic Resonance Imaging) är en teknik som används inom sjukvården för att skapa detaljrika bilder av vävnad och organ hos människor och djur på ett ofarligt och icke-påträngande sätt \parencite{mri_nobelpris_pressmeddelande}. MRI-scannern som används i sjukvården består av en cylinder med ett skjutbart bord i som patienten ligger på, se figur \ref{fig:mri_patient}. 

Till skillnad från tidigare tekniker för diagnostisk avbildning såsom röntgen, används ingen joniserande strålning i MRI-undersökningar. Istället används ett mycket starkt magnetfält och radiovåg-pulsar. Tekniken bygger på konceptet av kärnmagnetisk resonans/NMR (Nuclear Magnetic Resonance), vars teori hade sina rötter i upptäckten av protonens spinnegenskaper under 1920-talet \parencite{mri_lärobok}.

\begin{figure}[ht]
	\centering
	\includegraphics[width=.7\textwidth]{mri_patient}
	\caption{En patient som genomgår en MRI-undersökning. Foto av MART PRODUCTION \parencite*{fig:mri_patient} från Pexels.}
	\label{fig:mri_patient}
\end{figure}

Benämningen NMR används främst för det fysikaliska fenomenet och inte längre i sammanhang av diagnostiska undersökningar av individer \parencite{nmr_eller_mri}. \say{Magnetisk resonans} är mindre specifikt än \say{kärnmagnetisk resonans} eftersom det även finns elektronspinnresonans som involverar elektronen och inte atomkärnan. Ordet \say{nuclear} eller prefixet \say{kärn-} förknippas med farliga saker av allmänheten, så man undviker ofta att använda den mer specifika benämningen trots att det bara syftar på att man studerar atomkärnan.

Kärnmagnetisk resonans används även inom kemin för att undersöka strukturen hos olika ämnen \parencite{nmr_kemi}. Då produceras ett NMR-spektra, istället för en 2D- eller 3D-bild, som avslöjar egenskaper hos ämnet. År 1945 utvecklade Felix Bloch och Edward Mills Purcell, med kunskap om kvantmekaniken som utvecklades under 1930-talet och tidigare, de första mätningarna som använde sig av kärnmagnetisk resonans \parencite{mri_lärobok}. De mätte då signaler från ett vattenprov och ett paraffinprov, och förklarade de experimentella och teoretiska detaljerna av NMR. För detta delade Bloch och Purcell nobelpriset i fysik år 1952 \parencite{nmr_nobelpris}.

Två av de främsta utvecklarna av MRI-teknologin var Paul Lauterbur och Peter Mansfield \parencite{mri_nobelpris_pressmeddelande}. Under 1970-talet gjorde Lauterbur och Mansfield flera upptäckter som lade en grund för utvecklingen av MRI. 1973 beskrev Lauterbur hur gradientmagneter kunde användas tillsammans med NMR för att göra flera endimensionella projektioner av ett objekt, som sedan sätts ihop till en två- eller tre-dimensionell bild \parencite{lauterbur_image_formation}. I hans experiment kunde han använda tekniken för att skilja på vanligt och tungt vatten, något som ingen annan avbildningsmetod kunde göra. Samma år utnyttjade Mansfield tillsammans med Peter Grannell gradientmagneter med NMR för att undersöka strukturen i fasta ämnen på ett mer exakt sätt \parencite{mansfield_solid_structure}. 1977 tog Mansfield fram en ny metod för NMR-avbildning som utnyttjade egenskaperna hos en effekt som kallas spinn-eko \parencite{mansfield_fast_image_formation}. Denna nya metod kunde producera bilder snabbare än tidigare metoder. Lauterbur och Mansfield fick år 2003 nobelpriset i Fysiologi eller Medicin för sina bidrag till MRI-tekniken \parencite{mri_nobelpris_pressmeddelande}.

Magnetisk resonanstomografi började tillämpas i sjukvården redan i början av 1980-talet, och år 2002 fanns det mer än 22 000 magnetkameror i hela världen \parencite{mri_nobelpris_pressmeddelande}. Utvecklingen av MRI revolutionerade medicinen, och sedan dess har tekniken fortsatt att utvecklas och förfinas \parencite{mri_facts}. Magnetkameran används för att undersöka alla organ i kroppen. Några exempel på problem som undersöks med magnetkameror är abnormaliteter i hjärnan och ryggraden, tumörer och cystor i olika delar av kroppen, bröstcancer, skador i lederna, och hjärtproblem. En annan användning av MRI kallas fMRI (functional Magnetic Resonance Imaging), och innebär att man mäter kognitiv aktivitet i hjärnan genom att undersöka blodflödet i olika delar av hjärnan. Det ger viss information om neuronernas aktivitet även om det inte är vad man direkt mäter.

\clearpage
\section{Utrustning}
Till att börja med kan det vara bra att införa ett koordinatsystem för att benämna olika axlar i MRI-systemet. Riktningen som går från patientens fötter till huvud är Z-axeln. X-axeln går från patientens högra till vänstra sida och Y-axeln går upp från marken, rygg till mage. Det här är ett vanligt koordinatsystem som används inom MRI \parencite{mri_for_radiologists}.

\begin{figure}[hp]
	\centering
	\includegraphics[width=.8\textwidth]{mri_schematic}
	\caption{Diagram av en magnetkamera och placeringen av de olika spolarna. Bildkälla: \cite{fig:mri_spolar_diagram}.}
	\label{fig:mri_diagram}
\end{figure}

Figur \ref{fig:mri_diagram} visar ett diagram av de primära komponenterna i en magnetkamera. Magnetkamerans största beståndsdel är den mycket starka elektromagneten, placerad längst ut från cylinderns center. Elektromagneten består av en solenoidspole som bildar ett homogent magnetfält $\vec{B_0}$ i centret där patienten befinner sig under undersökningen. $B_0$ brukar vara 1.5 T eller 3 T för kliniskt bruk \parencite{understanding_mri}. Figur \ref{fig:solenoid} visar en sådan spole, där de raka linjerna motsvarar fältet $\vec{B_0}$ som går längs Z-axeln i MRI-systemet. För att uppnå den höga fältstyrkan på ett effektivt sätt används en supraledande metallegering i spolen, som kyls ner till runt 4 K med hjälp av flytande helium. Det gör att strömmen kan flöda med minimal resistans och därmed att den magnetiska flödestätheten blir större för samma effekt, enligt $B_0\propto\sqrt\frac{P}{R}$. Resistansen i supraledaren är dock så låg att detta samband inte beskriver systemet korrekt. Magneten kan hålla kvar strömmen helt utan pålagd spänning i flera år efter en första upprampning \parencite{mri_for_radiologists}.

\begin{figure}[ht]
	\centering
	\includesvg{Solenoidspole}
	\caption{En solenoidspole som kan liknas med den starka elektromagneten i MRI-systemet. Bilkälla: \cite{fig:solenoid}.}
	\label{fig:solenoid}
\end{figure}

Närmare centret av magnetkamerans cylinder finns tre gradient\-magnet-spolar (se figur \ref{fig:mri_diagram}) \parencite{understanding_mri}. Dessa har syftet att variera magnetfältet linjärt över valfri tredimensionell riktning. Gradientmagneterna motsvarar de tre axlarna och deras styrkor kan justeras dynamiskt för att ändra det resulterande fältets riktning. Eftersom endast en tredimensionell relativ lutning av fältstyrka behövs är gradientmagneterna relativt svaga och är inte nedkylda. Gradientmagneterna justerar bara riktningen av det starka magnetfältet som kommer från den primära elektromagneten.

Den tredje uppsättningen spolar är RF(Radio Frequency/radiofrekvens)-spolarna, som ligger ännu närmare patienten \parencite{understanding_mri}. Syftet med dessa är att skicka energi i form av elektromagnetiska vågor med en viss (icke-joniserande) frekvens i MHz-området (radiovågor) in i vävnaden, samt att detektera emitterade radiovågor från vävnaden. Radiofrekvens-fältet benämns $\vec{B_1}$ och kan även det antas vara homogent (därmed vektor-representationen). $\vec{B_1}$ ligger vinkelrätt mot $\vec{B_0}$. För fMRI används en separat RF-spole vid huvudet för att maximera signalerna från hjärnan. Eftersom det finns många andra källor av radiovågor (exampelvis från radio\-kommunikation) som når ut till där magnetkameran befinner sig, behöver rummet vara en Faradays bur. På så sätt kontaminerar inte de yttre och de inre radiovågorna varandra.

En annan mycket viktig komponent av MRI-systemet är datorerna som behandlar all data och som ger operatören ett grafiskt användargränssnitt att styra med och se resultaten på. Dessa är placerade i ett annat rum, kontrollcentret där operatören sitter. Datan som samlas in av magnetkameran transformeras med komplexa signalbehandlings-algoritmer för att producera de resulterande bilderna.

\clearpage
\section{Kvantmekaniska principer}
I magnetisk resonanstomografi studeras kärnan hos väte, protonen. Det är delvis för att människor innehåller så mycket väte, främst i vattnet som vi består av till runt 66\% \parencite{mri_nobelpris_pressmeddelande}. Men det är också på grund av protonens kvantmekaniska egenskaper. Härefter kommer en bakgrund till de kvantmekaniska principerna som MRI bygger på.

\subsection{Spinn}
Subatomära partiklar har en kvantmekanisk egenskap som kallas spinn \parencite{college_physics}. Spinnkvanttalet $s$ bestämmer magnituden av ett inre rörelsemängdsmoment $\vec{S}$ som är speciellt för varje typ av partikel. Man kan tänka sig att partikeln roterar runt en egen axel på samma sätt som en boll kan göra i klassisk fysik. Den liknelsen håller dock inte hela vägen, och den är extra orimlig för just elektroner - den hypotetiska ytan hade behövt röra sig med en fart snabbare än ljusets för att det inre rörelsemängdsmomentet ska vara korrekt \parencite{electron_spin}. 

\say{spinn} används här som synonym till $\vec{S}$, alltså \say{inre rörelsemängds\-moment hos en subatomär partikel}, och är en vektor. I klassisk fysik bestäms rörelse\-mängdsmomentet hos en partikel av kryssprodukten $\vec{L}=m\vec{r}\times\vec{v}$, där $\vec{r}$ är partikelns position relativt rotationens center och $\vec{v}$ är hastighetsvektorn vid samma tidpunkt. Man kan tänka sig att vår subatomära partikel består av mindre partiklar. Det totala rörelsemängdsmomentet hos dessa har då samma riktning som spinnet, alltså uppåt om den roterar motsols i planet sett uppifrån. Magnituden av spinnet bestäms av $S=\hbar\sqrt{s(s+1)}\:\left[\si{kg.m^2.s^{-1}}\right]$ (där $\hbar=\frac{h}{2\pi}$) \parencite{college_physics}. $s$ förekommer kvantiserat i steg av $\frac{1}{2}$ hos olika partiklar, vilket innebär att $S$ och $\vec{S}$ också är kvantiserat. Protoner, neutroner och elektroner har $s=\frac{1}{2}$ medan vissa andra partiklar har $s=1$ eller $s=0$. Partiklar med heltals-spinn (inklusive 0) klassas som fermioner och partiklar med halvtaligt spinn är bosoner \parencite{subatomic_particles}.

\begin{figure}[p]
	\centering
	\includegraphics[width=.5\textwidth]{snurra_precession}
	\caption{Precession av en snurra, där $L$ är rörelsemängdsmoment-vektorn, $\omega_p$ är precessions-frekvensen och $\omega_s$ är spinnfrekvensen. Vridmomentet $\tau$ bildas av gravitationskraften $F_g$ och den reaktiva normalkraften $-F_g$ vid bordets yta. Riktningen av detta vridmoment vrids på grund av det egna spinnet och gör att snurran precesserar. Bildkälla: Xavier Snelgrove \parencite*{fig:snurra_precession}, CC BY-SA 2.5, via Wikimedia Commons.}
	\label{fig:snurra_precession}
\end{figure}
\begin{figure}[p]
	\centering
	\includegraphics[width=.5\textwidth]{magnetic_moment_precession}
	\caption{En partikel med ett kvantspinn som gör att det magnetiska dipolmomentet $\vec\mu$ bildas, placerat i ett yttre magnetfält $\vec B_0$ som gör att dipolmomentet precesserar likt snurran. Källa: \cite{mri_lärobok}.}
	\label{fig:magnetiskt_moment_precession}
\end{figure}

\subsection{Precession}
Partiklar som både har en laddning och ett spinn $S\neq0$ beter sig som en liten magnet, med ett magnetiskt dipolmoment $\vec{\mu}$ riktat i samma riktning som $\vec{S}$ om laddningen är positiv. Detta kan liknas med en elektromagnet, där negativt laddade elektroner flödar i en cirkulär rörelse och därmed bildar ett magnetiskt dipolmoment med motsatta riktningen av elektronens rörelsemängdsmoment. Dipolmomentets magnitud definieras klassiskt som strömmen i spolen gånger cirkelns area, $\mu=IA=\pi Ir^2$, och har enheten \si{\left[A.m^2\right]} \parencite{magnetism}. Dipolmomentet hos en subatomär partikel relateras till spinnet genom $\vec{\mu}=\gamma\vec{S}$, där $\gamma$ är den gyromagnetiska kvoten för partikeln \parencite{larmor_precession}, som vi återkommer till. $\gamma$ är \SI{2.675 221 900(18) e8}{\left[s^{-1}.T^{-1}\right]} för protoner \parencite{gyro_ratio}. 

När en partikel med en egen magnetisk dipol dessutom placeras i ett homogent yttre magnetfält $\vec{B_0}$ så börjar dipolmomentet precessera runt det likt hur en vanlig leksakssnurra precesserar runt ett gravitationsfält \parencite{larmor_precession}, se figur \ref{fig:snurra_precession} och \ref{fig:magnetiskt_moment_precession}. Detta leder till två saker:
\begin{itemize}
	\item Spinnet $\vec{S}$ har en viss vinkel ($\theta$ i figur \ref{fig:magnetiskt_moment_precession}) mot Z-axeln. Projektionen av $\vec{S}$ längs Z-axeln bestäms av $S_z=m_s\hbar$, där $m_s$ är kvanttalet för spinnets projektion och är kvantiserat till värdena $m_s=s+n$ där $n\in\{-2s..0\}$ \parencite{university_physics}.
	\item $\vec{\mu}$ precesserar med en viss vinkelfrekvens som kallas Larmorfrekvensen. Även denna bestäms med hjälp av den gyromagnetiska kvoten, i Larmorekvationen \parencite{mri_lärobok}:
	\begin{equation}\label{eq:larmor}
		\omega=2\pi f=\gamma B_0\:[\si{rad/s}]
	\end{equation}
	Frekvensen ökar alltså linjärt med magnetfältetets styrka.
\end{itemize}

Hos protonen, likt elektronen, är $s=\frac{1}{2}$ och $m_s=\pm\frac{1}{2}$ \parencite{college_physics}. Vinkeln av spinnet mot Z-axeln blir då $\theta_z=\arccos\sqrt\frac{1}{3}\approx54.7^\circ$. Eftersom $s$ alltid har samma värde för elektroner brukar spinnkvanttalet utelämnas från kvanttalen, eller syfta på $m_s$. $m_s=\frac{1}{2}$ innebär att $\vec{\mu}$ precesserar parallellt med $\vec{B_0}$ och $m_s=-\frac{1}{2}$ innebär att det precesserar antiparallellt mot fältet. Det parallella tillståndet har något lägre potentiell energi än det antiparallella tillståndet eftersom det tar ett arbete för att motverka magnetfältet \parencite{electron_spin}. Spinnprojektionstalet $m_s$ kan ändras hos en partikel eftersom det motsvarar olika energinivåer, medan magnituden $s$ är konstant. 

\clearpage
\section{Magnetkamerans funktion}
Nu ska dessa kvantmekaniska principer sättas in i sitt sammanhang i magnetkameran. Först beskrivs hur signaler fångas upp från kroppen och hur de påverkas av egenskaper i vävnaden. Sedan beskrivs kortfattat hur signalerna kan lokaliseras till olika delar av personen för att skapa en bild. Fokus ligger snarare på fysiken runt NMR och inte på avbildningsdelen, eftersom den delen är mycket teknisk och handlar mer om matematisk signalbearbetning och algoritmer än om fysik.

\subsection{Magnetisering och excitering}
Vi börjar med att bortse från gradientmagneterna. När en människa placeras i magnetkameran utsätts alla protonerna i kroppen för det magnetiska fältet $\vec{B_0}$ riktat från fötterna till huvudet på personen \parencite{understanding_mri}. Det gör att ungefär hälften av protonernas magnetiska moment precesserar i samma riktning som magnetfältet och hälften antiparallellt. Eftersom de antiparallella har en något högre energinivå kommer det dock finnas ett litet underskott av dessa. För 1 miljon protoner som precesserar antiparallellt finns det ungefär 10-20 stycken fler\footnote{Från egna beräkningar. Siffran i artikeln av \textcite{understanding_mri} är tagen från boken \citetitle{mri_made_easy} av \textcite{mri_made_easy}, och är inte helt rimlig för parametrarna i en magnetkamera.} som precesserar parallellt med $\vec{B_0}$. Protonerna precesserar med (nästan) samma frekvenser, men eftersom de har slumpmässig fasförskjutning kommer det totala magnetiska dipolmomentet $\vec{M}$ i personen vara riktat i precis samma riktning som $\vec{B_0}$, vilket illustreras i figur \ref{fig:spinn_vektorer}. Personens totala magnetisering i detta tillfälle kallas också för den longitudinella magnetiseringen.

\begin{figure}[ht]
	\centering
	\includesvg[width=0.67\textwidth]{spinn_vektorer.svg}
	\caption{Illustrativt diagram av det magnetiska momentet hos olika protoner i kroppen av personen i magnetkameran. Vinklarna är skalenliga men inte längderna eller antalen. Rosa vektorer är individuella $\vec{\mu}$ hos olika protoner vid en viss tidpunkt, med något fler i parallellt spinntillstånd än antiparallellt. Blå vektor är hela personens magnetiska moment $\vec{M}=\sum{\vec{\mu_n}}$ (longitudinell magnetisering) och har därmed samma riktning som $\vec{B_0}$. Figuren är originell.}
	\label{fig:spinn_vektorer}
\end{figure}

$\vec{M}$ kan inte mätas eftersom den ligger i riktning med $\vec{B_0}$ \parencite{understanding_mri}. För att få information om vävnaden och dess struktur använder man sig av de tidigare nämnda RF-spolarna. Strömmen i en RF-spole varierar med en frekvens vilket gör att magnetfältet som bildas av spolen oscillerar med den frekvensen; elektromagnetiska vågor sänds ut \parencite{mri_for_radiologists}. När frekvensen är inställd till Larmorfrekvensen, alltså den som protonernas egna magnetfält precesserar med, så sker magnetisk resonans. Det innebär två saker \parencite{understanding_mri}:
\begin{itemize}
	\item Protoner kan ta upp energin $\Delta E=hf=\hbar\gamma B_0$ ($f$ från Larmorekvationen (\ref{eq:larmor})) och gå från det parallella spinntillståndet till det antiparallella tillståndet.
	\item Alla protonernas magnetiska moment börjar precessera i fas med varandra och de elektromagnetiska vågorna, så att $\vec{M}$ precesserar på samma sätt som en enskild proton.
\end{itemize}

Förresten, varför är inte alla protoner i det parallella tillståndet om det tillståndet har en lägre energinivå? Om protonen betedde sig som en vanlig magnet hade alla lagt sig i riktning med magnetfältet. Eftersom vi nu vet energiskillnaden mellan de två spinntillstånden kan vi förklara varför det alltid är just den andelen parallella och antiparallella protoner, med hjälp av Boltzmannfaktorn \parencite[s. 90]{mri_lärobok}:
\begin{equation}
	\frac{n_d}{n_u}=e^\frac{-\Delta E}{kT}=e^\frac{-\hbar\gamma B_0}{kT}
\end{equation}
Där $n_d$ är antalet protoner i det antiparallella spinntillståndet och $n_u$ är antalet i det parallella tillståndet. $k$ är Boltzmanns konstant och T är temperaturen. Ekvationen anger alltså relationen mellan sannolikheten av tillståndet med högre energi och sannolikheten av tillståndet med lägre energi, som en funktion av temperaturen och skillnaden i energi mellan de två tillstånden. För $T=\SI{310}{K}$ (kroppstemperatur), $B_0=\SI{1.5}{T}$, och $n_d=\SI{1000000}{}$ ger detta att $n_u\approx\SI{1000010}{}$ protoner. Orsaken till att nästan hälften av protonerna är i det högre energitillståndet är alltså delvis att energiskillnaden mellan tillstånden är så liten (\SI{5.3e-7}{eV}). Om det var normala magneter hade skillnaden i potentiell energi mellan de två riktningarna varit mer än enorm i jämförelse. Andelen protoner i den högre energinivån hade också varit låg om temperaturen i personen var väldigt nära den absoluta nollpunkten.

Hur många av protonerna som övergår till det antiparallella tillståndet på grund av radiovåg-pulsen beror på styrkan och varaktigheten av den \parencite{mri_for_radiologists}. Det innebär att vinkeln mot Z-axeln $\theta_z$ av $\vec{M}$ går från $0^\circ$ till en ny vinkel som är större ju fler protoner som tagit upp energi från pulsen. Detta går även att modellera som att $\vec{M}$ börjar precessera runt RF-fältet $\vec{B_1}$, vilket betyder att $\theta_z$ efter pulsen kan beräknas med Larmorekvationen (\ref{eq:larmor}) \parencite[s. 45]{mri_lärobok}:
\begin{equation}
	\theta_z=\omega t=\gamma B_1t
\end{equation}

Här betraktar vi en RF-puls som gör att $\vec{M}$ ligger i XY-planet precis när pulsen stängs av, en så kallad $90^\circ$-puls. Andra pulsar används också, exempelvis $180^\circ$-pulsar där den resulterande magnetiseringen precesserar antiparallellt \parencite{mri_for_radiologists}.

\subsection{Relaxation}
När pulsen är avslagen börjar protonerna frigöra överskottsenergin och därmed återgå till sina ursprungliga energinivåer \parencite{understanding_mri}. På samma gång hamnar protonernas dipolmoment återigen ur fas med varandra. Denna process kallas relaxation och sker över tid så att $\vec{M}$ rör sig på ett sätt som illustreras i figur \ref{fig:relaxation}. Eftersom $\vec{M}$ nu varierar gör det att det magnetiska flödet i RF-spolarna också varierar, och en spänning induceras enligt $U=-N\frac{\mathrm{d}\Phi}{\mathrm{d}t}$ där $N$ är antalet varv i spolen och $\Phi$ är det magnetiska flödet (Faradays induktionslag). På så sätt kan nu systemet känna av en MR-signal.

\begin{figure}[ht]
	\centering
	\includegraphics[width=0.7\textwidth]{relaxation}
	\caption{Rörelsen av $\vec{M}$ under relaxationen efter en $90^\circ$-RF-puls. Källa: \cite{mri_lärobok}.}
	\label{fig:relaxation}
\end{figure}

\subsubsection{T1-relaxation}
Relaxationen delas upp i två delar som orskas av olika processer, benämnda T1 och T2 \parencite{understanding_mri}. T1-relaxation syftar på återställningen av den longitudinella magnetiseringen, alltså hur projektionen av $\vec{M}$ på Z-axeln, $M_z$, förändras över tid direkt efter RF-pulsen. T1-relaxationen beror bara på att en mycket liten andel protoner återgår till sin lägre energinivå där de precesserar parallellt med $\vec{B_0}$ istället för antiparallellt. T1-relaxationens graf är en exponentiellt avtagande funktion som går från $0$ till den ursprungliga longitudinella magnetiseringen $M_{z0}$. T1-tiden ($T_1$) bestämmer hur lång tid relaxationen tar. Eftersom det inte går att säga exakt när den är färdig bestämmer $T_1$ istället hur lång tid det tar för $M_z$ att gå från 0 till $1-\frac{1}{e}\approx63\%$ av sitt slutliga värde $M_{z0}$. Personens totala magnetiska dipolmoment i Z-led efter RF-pulsen kan alltså beskrivas av funktionen \parencite{t1_relaxation}: 
\begin{equation}
	M_z(t)=M_{z0}(1-e^{-\frac{t}{T_1}})	
\end{equation}
T1-tiden beror på flödestätheten $B_0$ samt hur snabbt molekylerna runt protonen rör sig. Molekylernas rörelse genererar varierande magnetfält som påverkar protonen \parencite{understanding_mri}. Om molekylerna rör sig med frekvenser nära protonens Larmorfrekvens är T1-tiden kortare eftersom det leder till att protonen mer effektivt avger energi. Eftersom Larmorfrekvensen ökar linjärt med $B_0$ påverkar även magnetfältets styrka $T_1$. Generellt ger en starkare magnet en längre T1-tid \parencite{t1_relaxation}. Olika molekyler har olika rörelsefrekvens och därmed är $T_1$ olika för olika ämnen i kroppen \parencite{understanding_mri}. Till exempel är $T_1$ för obundet vatten längre än för delvis bundet vatten. Fett brukar ha en kort T1-tid.

\subsubsection{T2- och T2*-relaxation}
T2-relaxation är processen där magnituden av $\vec{M}$ projicerat i XY-planet minskar över tid efter pulsen \parencite{understanding_mri}. Med andra ord är T2 avklingandet av den transversella, roterande magnetiseringen. Denna process orsakas bara av att precessionen av protonerna hamnar ur fas med varandra igen efter resonansen (och medför inte någon energiöverföring, till skillnad från T1). Det finns två orsaker till denna desynkronisering:
\begin{enumerate}
	\item Protonernas egna magnetfält påverkar varandra och orsakar små fluktuerande variationer i Larmorfrekvensen hos individuella protoner enligt Larmorekvationen. Svängningar med olika frekvens hamnar ur fas över tid.
	\item Om $\vec{B_0}$-fältet inte är helt homogent så leder även det till att protonerna får små skillnader i Larmorfrekvens. Eftersom det är en konstant effekt är det möjligt att motverka den.
\end{enumerate}
T2* benämner de sammanlagda effekterna av 1. och 2., medan T2 endast benämner effekterna av 1. (vilket är de effekterna man vill mäta). För tillfället bortser vi från effekterna av 2. T2-tiden ($T_2$) bestämmer hur lång tid det tar för magnituden av det transversella magnetiska dipolmomentet att gå från det maximala värdet $M_{xy}(0)$ till $\frac{1}{e}\approx37\%$ av $M_{xy}(0)$. Alltså beskrivs den transversella magnituden efter RF-pulsen av funktionen \parencite[s. 59]{mri_lärobok}:
\begin{equation}\label{eq:t2_magnitude}
	M_{xy}(t)=M_{xy}(0)e^{-\frac{t}{T_2}}
\end{equation}
Protonernas påverkan på varandra som orsakar T2 är slumpmässigt fluktuerande men har en sammanlagd effekt som beror på ämnets egenskaper \parencite{understanding_mri}. T2-tiden är kortare ju mer protonerna i vävnaden interagerar med varandra. Om molekylerna är längre ifrån varandra blir T2 en längre process. Till exempel är $T_2$ för vätskor längre än för vatten som är bundet i vävnad, där molekylerna kan påverka varandra mer. T2 är en mycket snabbare process än T1 för mänsklig vävnad; $T_2\leq T_1$.

\subsection{Inducerad signal}
MR-signalen varierar som sagt på grund av variationen i $\vec{M}$ \parencite{understanding_mri}. Det medför att signalen kan modelleras som en dämpad svängning som representerar derivatan av den transversella svängningen projicerat på axeln av $\vec{B_1}$ (som är vinkelrät mot Z-axeln), enligt Faradays induktionslag. Signalen från en sådan svängning kallas Free Induction Decay (FID), och illustreras i figur \ref{fig:fid}. En FID-signal innehåller bara information om den transversella magnetiseringens förändring i en riktning, men hur gör man då för att hitta $T_1$? För att göra det behöver man två $90^\circ$-pulsar. En liten tid efter den första pulsen kommer $\vec{M_z}$ vara delvis återställd, och då appliceras en till puls som vrider denna vektor ner till transversalplanet. Då kommer $M_{xy}$ vara proportionell mot den nya longitudinella magnetiseringen, och man får två värden på $M_z$ som sedan kan användas för att bestämma $T_1$. En lång tid mellan pulsarna ger mindre kontrast mellan olika T1-värden medan en kort repetitionstid ger större kontrast.

\begin{figure}[ht]
	\centering
	\includesvg[width=0.7\textwidth]{fid}
	\caption{En graf som visar den inducerade spänningen i en RF-spole efter en absorberad $90^\circ$-puls. Spänningen oscillerar med Larmorfrekvensen och dämpas enligt ekvation (\ref{eq:t2_magnitude}). Bildkälla: \textcite{fig:fid}, CC BY-SA 3.0, via Wikimedia Commons.}
	\label{fig:fid}
\end{figure}

$T^*_2$ kan beräknas ur FID-signalen med en enda RF-puls, men $T^*_2$ varierar som sagt inte bara beroende på vävnadets egenskaper. Som nämnt tidigare går det att motverka effekten av de små inhomogeniteterna i $\vec{B_0}$-fältet som ger en avvikelse från $T_2$. Här kommer konceptet av spinn-eko in \parencite{understanding_mri}. Om man applicerar en $180^\circ$-puls efter att protonerna har börjat hamna ur fas, så byter protonernas precession riktning. Protonerna som innan precesserade motsols kommer nu börja precessera medsols, och tvärtom. Det gör att de protoner som har en konstant högre Larmorfrekvens återigen kommer i fas med protonerna med konstant lägre Larmorfrekvens, och $M_{xy}$ ökar i amplitud innan den minskar igen. $180^\circ$-pulsar appliceras med regelbundna mellanrum, och återfasningarna ger nya värden på T2-signalen med inhomogeniteten motverkad. Då kan det äkta värdet på $T_2$ beräknas, som alltid kommer vara större än $T^*_2$. Tiden mellan $180^\circ$-pulsarna kan justeras som en till parameter och påverkar bland annat signalens styrka.

\subsection{Dimensionell avbildning}
Magnetisk resonanstomografi går ut på att göra bilder bestående av pixlar med ljusstyrkor som beror på $T_1$ eller $T_2$ i olika delar av personen \parencite{understanding_mri}. Det som har förklarats hittills är NMR; vi har inte ännu ett sätt att skapa bilder av MR-signalen. Till sist kommer gradientmagneterna in i bilden\footnote{Oavsiktlig ordvits}. Enligt Larmorekvationen varierar precessions-frekvensen och därmed FID-signalens frekvens linjärt med $B_0$. Gradientmagneterna skapar en linjär variation av magnetisk flödestäthet genom personen, vilket medför en linjär variation av Larmorfrekvenser. Det gör att man kan hitta relaxationsvärden för olika skivor av kroppen. 

För att göra en endimensionell avbildning behöver man samla in signaler för varje Larmorfrekvens, men med hjälp av fouriertransformen kan man samla in en enda signal med alla frekvenser på samma gång \parencite{mri_lärobok}. Istället för en RF-puls med en enda frekvens används en puls som innehåller alla frekvenser upp till en viss gräns. Sinc-funktionen är en viktig funktion inom signalbehandling då den upfyller detta krav. Den diskreta fouriertransformen används för att dela upp den insamlade signalen i frekvens- och faskomposanter. För den som känner till basisvektorer är detta ekvivalent med att representera den diskreta signalen som en komplex vektor och ändra basisvektorerna till komplexa svängningar (eulers formel) med heltalsfrekvenser (dessa är ortogonala) \parencite{dsp_bok_kapitel}. För att göra bilder i två eller tre dimensioner används tre separata gradient-magnetfält, flera pulssekvenser och en mer invecklad signalavkodningsteknik \parencite{mri_lärobok}. 

Detta är en mycket översiktlig, konceptuell förklaring av avbildningstekniken som inte är helt korrekt men som ger en intuitiv idé om gradientmagneternas funktion. Det finns också många variationer och optimiserade tekniker som inte får plats att beskrivas här.

\clearpage
\section{Avslutning}
De grundläggande kvantmekaniska principerna som gör magnetkameran möjlig har förklarats och använts för att ge insikt i tekniken. Magnetisk resonanstomografi är ett mycket stort ämne. Som med de flesta breda ämnen är det som en fraktal av information som man kan kan undersöka på många nivåer av förstoring. Något som jag märkt när jag letat källor och läst olika texter om MRI är en segregation mellan väldigt konceptuella förklaringar riktade mot allmänheten och väldigt ingående förklaringar med långa härledningar riktade mot personer med högre utbildning. Därför var det en utmaning att hitta en förståelse som både utnyttjar kunskaper i lämplig nivå av fysik och som inte lämnar för stora hål.  Min förhoppning är att denna text gett en grundläggande, men någorlunda omfattande förståelse för magnetkamerans funktion ur ett fysikperspektiv.

%Magnetkameran är ett exempel på hur upptäckter inom fysiken direkt kan leda till betydelsefulla innovationer. 

\clearpage
\printbibliography

\end{document}
